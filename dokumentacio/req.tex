\section{Requirements}
\subsection{Installing Python 3}
\begin{enumerate}
\item Visit the official Python website at: \url{https://www.python.org/downloads/}.
\item Select the latest Python 3 version and click on the download link.
\item Run the downloaded file and follow the instructions to install Python on your system.
\item Verify the installation by running Python version in the command line or terminal:
\begin{lstlisting}[language=bash]
python3 --version
\end{lstlisting}
Your system should respond with the installed Python 3 version.
\end{enumerate}

\clearpage
\subsection{Installing Tkinter}
Tkinter is usually part of Python 3, so a separate installation is typically not required. If it is not installed on your system, you can install it with the following commands:
\begin{enumerate}
\item \textbf{For Debian/Ubuntu systems:}
\begin{lstlisting}[language=bash]
sudo apt-get update
sudo apt-get install python3-tk
\end{lstlisting}
\item \textbf{For Red Hat/Fedora systems:}
\begin{lstlisting}[language=bash]
sudo dnf install python3-tkinter
\end{lstlisting}
\item \textbf{For Windows systems:}
Tkinter is usually automatically installed with Python.
\end{enumerate}

\subsection{Installing Packages}
\subsubsection*{CustomTkinter}
\begin{lstlisting}[language=sh]
    pip install customtkinter
\end{lstlisting}

\begin{itemize}
\item \textbf{Ghostscript:} For saving diagrams.
\end{itemize}
\subsubsection*{For Windows systems:}
\begin{enumerate}
\item Download the Ghostscript installer from the official \href{https://www.ghostscript.com/download/gsdnld.html}{Ghostscript website}.
\item Run the installer and follow the installation instructions.
\item Add the directory where the \texttt{gs.exe} is located to the system PATH environment variable. This is usually \texttt{C:\textbackslash Program Files\textbackslash gs\textbackslash gsX.XX\textbackslash bin} (where \texttt{X.XX} is the version number).
\end{enumerate}

\subsubsection*{For macOS systems:}
Ghostscript can be installed using Homebrew, a package manager for macOS:
\begin{lstlisting}[language=sh]
brew install ghostscript
\end{lstlisting}

\subsubsection*{For Linux systems:}
Most Linux distributions can install Ghostscript through their package manager. For example, on Ubuntu or Debian-based systems:
\begin{lstlisting}[language=sh]
sudo apt-get update
sudo apt-get install ghostscript
\end{lstlisting}