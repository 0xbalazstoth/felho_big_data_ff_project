\section{Data collection and processing}
Main screen of the application, especially \textbf{Scrape} option.
\customwidthimage{main}{10cm}

\subsubsection*{Gépigény scraping}
\customwidthimage{gepigeny}{10cm}
\subsubsection*{Initializing the Scraper}
The \texttt{GepigenyScraper} class is designed to facilitate scraping comments from the Gepigeny website. Its constructor (\texttt{\_\_init\_\_} method) takes three parameters: \texttt{base\_url}, \texttt{save\_callback}, and \texttt{status\_label}. 

\begin{itemize}
    \item \texttt{base\_url}: This parameter represents the base URL of the Gepigeny website from which comments will be scraped.
    \item \texttt{save\_callback}: This parameter is a function that will be called to save the scraped data.
    \item \texttt{status\_label}: This parameter is a reference to a status label widget that will display information about the scraping process.
\end{itemize}

\subsubsection*{Scraping Gepigeny Comments}
The \texttt{scrape\_gepigeny\_comments} method of the \texttt{GepigenyScraper} class is responsible for actually scraping comments from the Gepigeny website.

\begin{itemize}
    \item It starts by retrieving the base URL from the \texttt{base\_url} attribute of the instance.
    \item Then, it sends a request to the base URL to retrieve the content of the page.
    \item It uses BeautifulSoup to parse the HTML content and extract the total number of pages of comments available.
    \item After that, it iterates over each page of comments, extracting the comments and appending them to a list.
    \item It saves the comments to a JSON file after each page.
    \item It also updates the status label with information about the scraping progress.
    \item Finally, it prints a message indicating the completion of the scraping process along with the total number of comments scraped.
\end{itemize}

\clearpage
\subsubsection*{Google Play scraping}
\customwidthimage{google_play}{10cm}
\subsubsection*{Initializing the Scraper}
The \texttt{GooglePlayScraper} class is designed for scraping reviews from the Google Play Store. Its constructor (\texttt{\_\_init\_\_} method) takes two parameters: \texttt{app\_id\_entry} and \texttt{save\_callback}.

\begin{itemize}
    \item \texttt{app\_id\_entry}: This parameter represents the Entry widget for app ID input.
    \item \texttt{save\_callback}: This parameter is a function that will be called to save the scraped data.
\end{itemize}

\subsubsection*{Scraping Google Play Reviews}
The \texttt{scrape\_reviews} method of the \texttt{GooglePlayScraper} class is responsible for scraping reviews for the specified app ID.

\begin{itemize}
    \item It retrieves the app ID from the \texttt{app\_id\_entry} widget.
    \item Then, it calls the \texttt{reviews\_all} function from the \texttt{google\_play\_scraper} library to scrape all reviews for the specified app ID.
    \item It specifies the language and country parameters as 'hu' (Hungarian) for the Hungarian version of Google Play.
    \item It saves the scraped reviews using the \texttt{save\_callback} function.
    \item If an error occurs during scraping, it displays an error message using a messagebox.
\end{itemize}

\subsubsection*{MarkMyProfessor scraping}
\customwidthimage{markmyprofessor}{10cm}
\subsubsection*{Initializing the Scraper}
The \texttt{MarkMyProfessorScraper} class is designed for scraping comments and ratings for teachers from the MarkMyProfessor website. Its constructor (\texttt{\_\_init\_\_} method) takes two parameters: \texttt{name\_entry} and \texttt{save\_callback}.

\begin{itemize}
    \item \texttt{name\_entry}: This parameter represents the Entry widget for teacher name input.
    \item \texttt{save\_callback}: This parameter is a function that will be called to save the scraped data.
\end{itemize}

\subsubsection*{Searching for Teachers}
The \texttt{search\_teachers} method of the \texttt{MarkMyProfessorScraper} class is responsible for searching for teachers based on the provided name query.

\begin{itemize}
    \item It retrieves the name query from the \texttt{name\_entry} widget.
    \item Constructs a URL for the search query to the MarkMyProfessor backend API.
    \item Sends a GET request to the constructed URL and retrieves the response.
    \item Parses the JSON response to extract information about the teachers matching the query.
    \item Calls the \texttt{save\_callback} function to save the retrieved teacher data.
\end{itemize}

\subsubsection*{Scraping MarkMyProfessor Comments}
The \texttt{scrape\_markmyprofessor\_comments} method of the \texttt{MarkMyProfessorScraper} class is responsible for scraping comments and ratings for teachers from the MarkMyProfessor website.

\begin{itemize}
    \item Prompts the user to select a JSON file containing teacher data.
    \item Parses the selected JSON file and extracts data about the teachers.
    \item Iterates over each teacher in the data.
    \item For each teacher, retrieves the slug (unique identifier) and the last page of comments.
    \item Iterates over each page of comments for the teacher, scraping the comments and saving them to JSON files.
\end{itemize}

\subsubsection*{Helper Functions}
The \texttt{get\_markmyprofessor\_teacher\_get\_last\_page} method is a helper function that retrieves the last page of comments for a given teacher slug from the MarkMyProfessor backend API.

\begin{itemize}
    \item Constructs a URL for the teacher's ratings page to the MarkMyProfessor backend API.
    \item Sends a GET request to the constructed URL and retrieves the response.
    \item Parses the JSON response to extract the last page number of comments.
    \item Returns the last page number.
\end{itemize}

The \texttt{scrape\_markmyprofessor\_comment} method is a helper function that scrapes comments for a given teacher slug and page number from the MarkMyProfessor backend API.

\begin{itemize}
    \item Constructs a URL for the teacher's ratings page to the MarkMyProfessor backend API.
    \item Sends a GET request to the constructed URL and retrieves the response.
    \item Saves the retrieved comments to a JSON file.
    \item Returns the JSON response.
\end{itemize}

\subsubsection*{Preprocessing data}
Select \textbf{Preprocess} option.
\customwidthimage{main}{10cm}

\subsubsection*{Data processing}
\begin{enumerate}
    \item Load data (csv file format)
    \item Options: Clean text, Remove duplicates, Remove numeric values
\end{enumerate}

\noindent \\
Important to mention that the clean text option cleans the data with these methods:
\begin{itemize}
    \item It converts text to lowercase to ensure consistency.
    \item It removes URLs using regular expressions to eliminate any web links present in the text.
    \item It removes mentions by matching and replacing patterns starting with '@'.
    \item It removes hashtags by matching and replacing patterns starting with '\#'.
    \item It removes extra whitespace by replacing consecutive whitespace characters with a single space.
    \item Finally, it removes emojis from the text using the demoji library to ensure that only text data remains.
\end{itemize}

\customwidthimage{preprocess}{10cm}