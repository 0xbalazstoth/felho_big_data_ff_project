\documentclass{article}

\usepackage[hungarian]{babel}
\usepackage{listings}
\usepackage{xcolor}
\usepackage{hyperref}

\lstdefinestyle{mystyle}{
    backgroundcolor=\color{white},
    commentstyle=\color{green},
    keywordstyle=\color{blue},
    numberstyle=\tiny\color{gray},
    stringstyle=\color{red},
    basicstyle=\footnotesize\ttfamily,
    breakatwhitespace=false,
    breaklines=true,
    captionpos=b,
    keepspaces=true,
    numbers=left,
    numbersep=5pt,
    showspaces=false,
    showstringspaces=false,
    showtabs=false,
    tabsize=2
}

% Set code formatting
\lstset{style=mystyle}

\date{}

\begin{document}

\title{Felhő alapú IoT és Big Data platformok\\ \small Dokumentáció}
\author{Tóth Balázs - MWZX0D}
\maketitle

\renewcommand{\contentsname}{Tartalomjegyzék}

\tableofcontents
\clearpage

\section{Bevezetés}
A dokumentum bemutatja a \textbf{Python} programozási nyelv és a \textbf{Tkinter} grafikus felhasználói felületi toolkit segítségével fejlesztett Big Data interaktív adatvizualizációs csomagot. A csomag célja lehetővé tenni a felhasználók számára, hogy könnyedén vizualizálják és manipulálják nagy méretű adathalmazokat interaktív módon. A csomagot könnyen integrálhatják különböző adatelemzési folyamatokba és alkalmazásokba, ami lehetővé teszi a felhasználók számára, hogy felfedezzék és megértsék az adatokat a vizualizációk segítségével.

\section{Követelmények}
A Big Data interaktív adatvizualizációs csomag használatához a következő követelmények teljesülésére van szükség:

\subsection{Python 3 Telepítése}
\begin{enumerate}
    \item Látogasson el a Python hivatalos webhelyére a következő címen: \url{https://www.python.org/downloads/}.
    \item Válassza ki a legfrissebb Python 3 verziót, és kattintson a letöltésre.
    \item Indítsa el a letöltött fájlt, és kövesse az utasításokat a Python telepítéséhez a rendszerére.
    \item Ellenőrizze a telepítést a Python verziójának futtatásával a parancssorban vagy terminálban:
    \begin{lstlisting}[language=bash]
    python3 --version
    \end{lstlisting}
    A rendszernek válaszolnia kell a telepített Python 3 verziójával.
\end{enumerate}

\subsection{Tkinter Telepítése}
A Tkinter általában a Python 3 része, így nincs szükség külön telepítésre. Ha mégsem lenne telepítve a rendszerére, akkor a következő parancsok segítségével telepítheti:
\begin{enumerate}
    \item \textbf{Debian/Ubuntu rendszerek esetén:}
    \begin{lstlisting}[language=bash]
    sudo apt-get update
    sudo apt-get install python3-tk
    \end{lstlisting}
    \item \textbf{Red Hat/Fedora rendszerek esetén:}
    \begin{lstlisting}[language=bash]
    sudo dnf install python3-tkinter
    \end{lstlisting}
    \item \textbf{Windows rendszerek esetén:}
    A Python telepítésekor a Tkinter általában automatikusan települ.
\end{enumerate}

\subsection{Szükséges Python Csomagok Telepítése}
A Big Data interaktív adatvizualizációs csomag használatához néhány további Python csomagot is telepíteni kell. Ezek a következők:
\begin{itemize}
    \item \textbf{numpy:} Matematikai műveletek végrehajtásához.
    \item \textbf{pandas:} Adatmanipulációhoz és -elemzéshez.
    \item \textbf{matplotlib:} Grafikonok és diagramok készítéséhez.
    \item \textbf{seaborn:} Adatvizualizációs könyvtár a matplotlib kiegészítéseként.
    \item \textbf{scikit-learn:} Gépi tanulás algoritmusokhoz és eszközökhöz.
\end{itemize}
Ezeket a csomagokat a Python pip csomagkezelőjével telepítheti:
\begin{lstlisting}[language=bash]
pip3 install numpy pandas matplotlib seaborn scikit-learn
\end{lstlisting}
Ez a parancs automatikusan letölti és telepíti a felsorolt csomagok legfrissebb verzióit a rendszerére.

\section{Használat}
Magyarázza el, hogyan kell használni a szoftvert itt.

\section{Kód Példák}
Itt vannak néhány kód példa:

\subsection{Python Példa}
\begin{lstlisting}[language=Python, caption=Python Példa]
def hello_world():
    print("Hello, world!")

hello_world()
\end{lstlisting}

\end{document}
